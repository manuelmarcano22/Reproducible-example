%%Example found at https://gist.github.com/Cadair/5f66da0c4d14055836b2

\documentclass[]{article}

\usepackage{pythontex}

\usepackage{graphicx}
\usepackage{pgf}
\usepackage{float}

%opening
\title{An Example of Astropy and PythonTeX Integration}
\author{Stuart Mumford}

\begin{pythontexcustomcode}{py}
import sys

import numpy as np

import matplotlib

matplotlib.use('pgf')

matplotlib.rc('text', usetex=True)
matplotlib.rc('font', family='serif')
matplotlib.rc('font', size=14.0)
matplotlib.rc('font', weight='normal')
matplotlib.rc('legend', fontsize=14.0)
matplotlib.rc("pgf", texsystem="pdflatex")

import matplotlib.pyplot as plt
from mpl_toolkits.axes_grid1 import make_axes_locatable

\end{pythontexcustomcode}

\begin{document}

\maketitle

\section{Basic PythonTeX}

The basic PythonTeX environments are \verb|\begin{pycode}| and \verb|\py|. \verb|pycode| is a block environment for including Python code, \verb|\py| is a inline output environment, for displaying output.

A simple example of both is to create a variable in a \verb|pycode| environment and then inline it in the text using \verb|\py|:

\begin{pyverbatim}
\begin{pycode}
import astropy.units as u

myquantity = 100 * u.kg/u.m**2
\end{pycode}
\end{pyverbatim}

\begin{pycode}
import astropy.units as u

myquantity = 100 * u.kg/u.m**2
\end{pycode}

You can then include this variable in a sentence using \verb~\py|myquantity|~. The density of Astropy is \py|myquantity|. Quantity has a little feature which is useful here: \verb~\py|myquantity_repr_latex_()|~ gives \py|myquantity._repr_latex_()|.


\end{document}
